
\documentclass[conference, letterpaper]{IEEEtran}

\usepackage{listings}

\hyphenation{op-tical net-works semi-conduc-tor}

% *** GRAPHICS RELATED PACKAGES ***
\ifCLASSINFOpdf
   \usepackage[pdftex]{graphicx}
\else
\fi

% *** MATH PACKAGES ***
\usepackage[cmex10]{amsmath}
\usepackage{color}
\usepackage{fancyhdr}
\usepackage[caption=false,font=footnotesize]{subfig}

\renewcommand{\thispagestyle}[2]{} 

\fancypagestyle{plain}{
        \fancyhead{}
        \fancyhead[C]{first page center header}
        \fancyfoot{}
        \fancyfoot[C]{first page center footer}
}
\pagestyle{fancy}

\headheight 31.99992pt
\footskip 20pt

\rhead{}

\setcounter{page}{1}

\fancyhead[R]{\textit{RIT Computer Science ~\textbullet~ Capstone Report ~\textbullet~ 2171}}
\renewcommand{\headrulewidth}{0pt}

%Footer
\fancyfoot[C]{Rochester Institute of Technology}
\renewcommand{\footrulewidth}{0.5pt}
\fancyfoot[R]{\thepage \  $|$ P a g e }

\begin{document}

\title{Using Transactional Events in Haskell in the Impelemntation of Message Broker Middleware}

\author{\IEEEauthorblockN{Christopher R. Wicks}
\IEEEauthorblockA{Department of Computer Science\\Golisano College of Computing and Information Sciences\\
Rochester Institute of Technology\\
Rochester, NY 14586\\
cw9887@cs.rit.edu}}

\maketitle

\begin{abstract}
Transactional events are used in the  implementation of message broker middleware. In particular, STOMP 
(Simple Text Oriented Messaging Protocol) client APIs, shared libraries, and a message broker are implemented 
utilizing an experimental transactional events library built on Concurrent Haskell.
\end{abstract}

\begin{IEEEkeywords}
Transactional Events, Concurrency, Haskell, Functional Programming, Monad, Message Broker, 
Message Oriented Middleware, STOMP
\end{IEEEkeywords}

\IEEEpeerreviewmaketitle

\section{Introduction}
Transactional events are a relatively recent concurrency abstraction in the functional programming
paradigm. They are a high-level abstraction that combine the first-class synchronous message-passing events
of Concurrent ML (CML) with all-or-nothing transactions \cite{te:original}. Their inception draws
inspiration from CML, Concurrent Haskell, and Software Transactional Memory (STM) Haskell. There currently 
exist implementations as language extensions for both Haskell \cite{te:original} and ML \cite{te:ml}.

Transactional events are primarily motivated by the need for better abstractions in the development of 
concurrent programs. The non-deterministic execution order of concurrent programs make them inherently
difficult to reason about. Transactional events help to ease some of the pain by allowing for inter-thread
communications to be composed of modular events in the form of transactions.

STOMP, the Simple (or Streaming) Text Oriented Message Protocol, is a text-based message passing protocol from the same school of design as HTTP.
\cite{stomp:spec}. It is designed so that applications in a distributed software system can communicate easily (from the 
perspective of an application developer) over a network.

``Every programming model needs three things: a semantics, an implementation, and useful idioms" 
\cite{te:idioms}. The semantics and the implementation have been provided in the work laid out in \cite{te:original}.
In this work, we explore the use of transactional events in the implementation and API design of a
collection of software tools and applications utilizing the STOMP protocol. We implement a client 
API, an command-line client application, a message broker (server), and shared libraries utilizing 
Transactional Events Haskell (TE Haskell). In particular, we identify useful idioms, patterns,
and novel use cases for transactional events as a programming model.

\section{Related Work}
To date there has been relatively little work around transactional events aside from the foundational work done by
Fluet and Donnelly in 2008 \cite{te:original}. Fluet and Amsden extend this foundation with a refined semantics  for ``fairness" in 2011 \cite{te:fairness}.
A detailed account of a large-scale implementation utilizing the abstraction does not appear to exist in the literature. 
In addition to Haskell, the semantics outlined in Fluet and Donnelly's 2008 work have been implemented in ML \cite{te:ml}. 
Kehrt, Effinger-Dean, Schmitz, and Grossman identify some basic problems that arise from some simple use cases for
transactional events, and propose idiomatic approaches to these issues in their work ``Programming Idioms for Transactional Events" \cite{te:idioms}.  

\bibliographystyle{IEEEtran}
\bibliography{IEEEabrv,citations}

\end{document}
